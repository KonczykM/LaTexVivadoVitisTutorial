\documentclass{article}
\usepackage{graphicx} % Required for inserting images
\usepackage{caption}
\usepackage{placeins} % For \FloatBarrier


\title{VivadoVitisGuide}
\author{Marcin Konczyk}
\date{June 2024}

% Define a new Template for steps
\newcounter{step}
\newcommand{\step}[3]{
    \refstepcounter{step}
    \subsection{Step \thestep: #1}
    \begin{enumerate}
        #2
    \end{enumerate}
    \begin{figure}[h!]
        \centering
        \includegraphics[width=1.1\textwidth]{images/step\thestep.png}
        \caption{#3}
        \label{fig:step\thestep}
    \end{figure}
    \FloatBarrier
    \newpage
}

\begin{document}
\tableofcontents
\newpage



\section{Importing Board Files}
% Step 1
\step{Downloading the Ebaz board files}{
    \item Download the Ebaz board files from \texttt{https://github.com/XyleMora/EBAZ4205/tree/main}.
    \item Transfer the board files `Ebaz4205` folder from the \texttt{download/Ebaz4205-main/Documents/Board files} to the Program folder specified during installation.
    \item The destination folder should look like this: \texttt{/Xilinx/Vivado/2023.2/data/boards/board\_files.}
    \item Place the board files into the destination folder at \texttt{/Xilinx/Vivado/2023.2/data/boards/board\_files.}
}{Download the Ebaz board files from \texttt{https://github.com/XyleMora/EBAZ4205/tree/main}.}

\section{Vivado work flow}
% Step 4
\step{Starting a new project in Vivado}{
    \item Open Vivado.
    \item Start a new project by clicking on Create Project.
    \item Specify the name.
    \item Specify your desired project directory.
}{Start a new project in Vivado.}

% Step 7
\step{Selecting the project type}{
    \item Ensure that RTL Project is selected.
    \item Choose Do not specify sources at this time.
}{Select the project type and proceed.}

\step{Board selection}{
    \item Select the board tab.
    \item In the Boards section, find the Ebaz4205 board by searching for it in the Search tab.
    \item Select the board (it should highlight blue).
    \item Proceed with project summary and finish project creation.
}{Board selection screen}

% Step 10
\step{Creating a block design}{
    \item You’re transported to the main screen of the project.
    \item From the workflow on the left of the screen, select “Create Block Design”.
    \item Name the block design or leave the default.
    \item Proceed
}{Create a block design in Vivado.}

% Step 12
\step{Adding a new IP block}{
    \item Now you should see the empty Diagram on your right.
    \item Click the small plus icon to Add New IP block.
    \item Search for the ZYNQ 7 Processing System.
    \item Double-click on the name to add it as a block.
}{Add the ZYNQ 7 Processing System block.}

% Step 14
\step{Running block automation}{
    \item You should now see the block has been created.
    \item At the top, you can see a hint to run block automation.
    \item Click on Run Block Automation.
    \item Ensure that the “Apply Board Preset” is selected.
    \item Click OK.
}{Run block automation for the ZYNQ 7 Processing System.}

% Step 16
\step{Configuring the ZYNQ 7 Processing System}{
    \item Now we can see that some lines were added to the outputs.
    \item All standard protocols have also been configured.
    \item Double-click on the ZYNQ 7 Processing System block to adjust its properties.
    \item You should be welcomed by the overview of the system.
    \item On the left, click on Peripheral I/O Pins.
}{Configure the ZYNQ 7 Processing System properties.}

% Step 18
\step{Configuring I/O pins}{
    \item You should be welcomed to the I/O configuration.
    \item Most of them should already be preconfigured (displayed as green).
    \item Disable the Ethernet 0 Protocol, as we won't be using it.
}{Configure the I/O pins and disable Ethernet 0 Protocol.}

% Step 19
\step{Adding more Block IPs}{
    \item Now we need to add more Block IPs.
    \item Click on the small plus icon and search for Processor System Reset.
    \item Double-click to add. It should appear on the screen.
    \item Connect the \texttt{FCLK\_CLK0} and \texttt{FCLK\_RESET0\_N} by drawing the line from one pin to the other.
    \item Connect the \texttt{FCLK\_CLK0} to the \texttt{M\_AXI\_GP0\_ACLK}.
}{Add and connect the Processor System Reset block.}

% Step 21
\step{Adding the AXI Interconnect block}{
    \item Add the next Block IP by clicking the small plus icon.
    \item Search for AXI Interconnect.
    \item Double-click to add.
    \item The AXI Interconnect Block should appear on your screen.
    \item Move blocks around by dragging them across the screen to make connections easier.
    \item Connect the Clock line \texttt{(FCLK\_CLK0)} and AXI GPIO line \texttt{(M\_AXI\_GP0)} from ZYNQ 7 Processing System to the AXI Interconnect as seen on the screen.
    \item Connect the \texttt{“interconnect\_aresetn”} from Processor System Reset to AXI Interconnect as seen on the screen.
    \item As you connect, you should see small green ticks indicating where you can (not always should) connect the lines.
}{Add the AXI Interconnect block.}

% Step 23
\step{Adding existing GPIO from the board}{
    \item After creating the AXI Interconnect, we need to add existing GPIO from the board.
    \item On your left tab, select the Board Tab at the top.
    \item You should see the EBAZ4205 Development Board.
    \item Right-click on existing GPIO (LEDs).
    \item Select Auto Connect.
    \item Click OK on the Auto Connect message.
}{Add existing GPIO from the board.}

% Step 26
\step{Connecting the AXI\_GPIO\_0 block}{
    \item This will create an \texttt{AXI\_GPIO\_0} block.
    \item Connect the block as seen on the screen.
    \item Connect the clock \texttt{(s\_axi\_aclk)} to \texttt{FCLK\_CLK0.}
    \item Connect reset (s\_axi\_aresetn) to the interconnect\_aresetn.
}{Connect the AXI\_GPIO\_0 block.}

% Step 27
\step{Completing the block design}{
    \item Your block design is now complete.
    \item Right-click on empty space on the diagram to show Diagram options.
    \item Click on Regenerate Layout for convenience.
}{Complete and clean up the block design.}

% Step 28
\step{Validating the design}{
    \item Your block design should now look cleaner and more concise.
    \item Click on Validate the design at the top (small check box icon).
    \item Save all changes and click on Auto Assign Addresses at the top.
}{Validate the block design.}

% Step 31
\step{Creating HDL wrapper}{
    \item To go back to the overview of the diagram, click the sources tab at the top.
    \item Now you should see all the sources in your design.
    \item If they are in the folder format, click the expand all button.
    \item Right-click on your main source file (should be the only one).
    \item Select Create HDL Wrapper option.
    \item Ensure that Let Vivado manage wrapper and auto-update is selected.
    \item Click OK.
}{Create the HDL wrapper.}

% Step 35
\step{Generating the Bitstream}{
    \item Wait until the Wrapper Creation is fully completed.
    \item Click the small green arrow with 1s and 0s at the top to Generate Bitstream.
    \item Confirm Bitstream generation
    \item Ensure the option says run on local host.
    \item Click OK (the number of jobs is not important and is to do with the CPU usage and generation time).
    \item Wait until the Bitstream is generated (might take a while).
    \item The Bitstream is fully generated once you receive the Bitstream Generation Completed message.
    \item Ensure that Open Implemented Design is selected.
    \item Click OK.
    \item You should now see a colorful system implementation diagram.
}{Generate the Bitstream.}

% Step 40
\step{Exporting the hardware}{
    \item Go to the File tab at the top.
    \item Select Export -> Export Hardware.
    \item You’re welcomed to the Hardware Platform Creation.
    \item Click Next.
    \item Ensure that the Include Bitstream option is selected.
    \item Click Next.
    \item Give a name to your platform or leave the default.
    \item Click Next.
    \item Rewiev and proceed with summary
}{Export the hardware configuration.}

% Step 45
\step{Launching Vitis IDE}{
    \item Go to the Tool tab.
    \item Select Launch Vitis IDE.
}{Launch Vitis IDE for software implementation.}

\section{Vitis Unfied IDE workflow}

% Step 46
\step{Accessing Vitis IDE and creating platform}{
    \item You’re welcomed to the Vitis IDE used for software implementation.
    \item Click on Create Platform Component in the Embedded Development section.
    \item Click on Create Platform Component in the Embedded Development section.
    \item Select the location you chose for your Vivado project.
    \item After the location is correctly selected, click Next.
}{Vitis IDE.}

% Step 51
\step{Selecting the hardware platform}{
    \item Click Browse to select your hardware platform that we generated from Vivado.
    \item Ensure that the file you chose has a .xsa extension.
    \item Click Next to proceed.
}{Select the hardware platform.}

% Step 54
\step{Configuring the operating system and processor}{
    \item If performed correctly, the operating system and processor should be automatically configured as seen.
    \item Click Next.
    \item Proceed with summary
    \item Click Finish.
}{Configure the operating system and processor.}

% Step 56
\step{Embedded Design Editor}{
    \item You should now be welcomed to the Embedded Design Editor.
    \item Your screen should look something like this.
    \item The platform should be the only existing file.
    \item If you don’t see the platform, repeat the platform creation from the tab on the left, by clicking Create Platform Component.
    \item On the left of your screen, select the Examples tab.
}{The Embedded Design Editor.}

% Step 58
\step{Creating an application component}{
    \item Find the example you desire (in our case it's Hello World).
    \item Select it and click on Create Application Component from Template.
    \item You can change the name of your application or leave the default.
    \item Click Next.
    \item Select the platform we just created (should be the only available platform).
    \item Click Next to proceed.
    \item Select the only existing domain.
    \item Click Next.
    \item Proceed with summary
    \item Click Finish.
}{Create an application component from a template.}

% Step 64
\step{Building application}{
    \item Now back in the Embedded Development tab, you should see an application.
    \item You can open the directories to see and modify the source file of your application.
    \item At the bottom left, select the Build option.
    \item Ensure that at the top, the component displays the name of your application.
    \item Click OK to build both the platform and application at the same time.
    \item You should now see a green tick mark next to the build option.
    \item This tick mark will disappear if you modify any of the source files, and you will have to rebuild the project again.
}{Overwiev of application source files.}

% Step 68
\step{Creating the boot image}{
    \item Click Create Boot Image as we are writing to an SD card.
    \item You should be welcomed by the Boot Image creation screen.
    \item We won't change anything and click Create Image.
}{Create the boot image.}

\step{Boot image location message}{
    \item After a moment, you should see the following message in your terminal.
    \item Remember the location of your boot image.
    \item It should look similar to the one on the screen.
}{Location of boot image in terminal}

% Step 70
\step{Finding the boot image}{
    \item Find the boot image in your file explorer.
    \item The path should look like this: \texttt{[project name]/[Application Name]/\_ide/bootimage.}
    \item In my case, \texttt{project\_2/hello\_world/\_ide/bootimage.}
    \item Place this file onto your SD card and put the SD card into the EBAZ4205 Development board.
    \item Connect the UART module to the computer and turn on the board.
}{Find the boot image and transfer it to the SD card.}


\end{document}